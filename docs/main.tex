\documentclass[conference]{IEEEtran}
\IEEEoverridecommandlockouts

\input{template}

\begin{document}

\title{Deep Learning Methods for ArUco Marker Detection and Classification Under Challenging Distortions}

\author{
\IEEEauthorblockN{
    Filip Hanuš\IEEEauthorrefmark{1},
}
\IEEEauthorblockA{
    \IEEEauthorrefmark{1}School of Engineering, College of Art, Technology and Environment,\\
    University of the West of England, Bristol, UK\\
    Email: filip2.hanus@uwe.ac.uk}
}

\maketitle
\begin{abstract}

Blabla

\end{abstract}

\begin{IEEEkeywords}
ArUco Markers, Deep Learning, Computer Vision, Object Detection, Image Classification, Fiducial Markers
\end{IEEEkeywords}

\section{Introduction}

ArUco markers are small, square visual codes or tags designed for easy detection and identification by computer vision algorithms. They are used for camera pose
estimation and localisation. Some examples of ArUco marker usage can be seen in Figure \ref{fig:aruco_markers}. For their use in these applications, 
they need to be tracked and/or recognised, which brings out the need for robust detection and classification methods.

\begin{figure}[h]
    \centering
    \begin{subfigure}[b]{0.45\textwidth}
        \centering
        \includegraphics[width=\textwidth]{images/aruco-example-1.png}
        \caption{Robot localisation (\cite{AI-Assisted-Drone-Localization})}
        \label{fig:aruco1}
    \end{subfigure}
    \hfill
    \begin{subfigure}[b]{0.45\textwidth}
        \centering
        \includegraphics[width=\textwidth]{images/aruco-example-2.png}
        \caption{Drone localisation (\cite{Nakajima2024AboutAE})}
        \label{fig:aruco2}
    \end{subfigure}
    \caption{Examples of ArUco markers in different conditions}
    \label{fig:aruco_markers}
\end{figure}

As \textcite{FiducialMarkerNoisy} point out, traditional detection methods perform quite well under controlled conditions, but less reliably when in real-world challenges,
such with rotations and noise. Building on that, this work explores deep learning approaches to enhance both detection and classification of ArUco markers under varying
challenging conditions. 

\section{Classification}

\subsection{Method: Classification}

Overview of classification approach

Data preparation and augmentation strategies
Selection and justification of network architectures (e.g., MinimalCNN, AlexNet, ResNet18, GoogLeNet)
Training configurations and hyperparameter choices

Algorithmic workflow (with diagrams/pseudocode)

Preprocessing steps
Model training loop
Validation and selection criteria

Key decisions and considerations

Handling distortions: rotations, blur, noise
Experimentation with multiple architectures and batch sizes

\subsection{Results: Classification}

Presentation of experimental setup

Datasets used (Files 2, 3, and custom datasets)
Evaluation metrics (accuracy, precision, recall, F1-score)

Numerical results and analysis

Comparison across different network architectures
Effect of parameters (learning rate, batch size, number of training images)
Graphs illustrating training accuracy, confusion matrices

Discussion of performance trends

Strengths and weaknesses observed
Cases of misclassification and potential causes

\section{Detection}

\subsection{Method: Detection}

Overview of detection approach

Data preparation: merging ArUco markers with office images
Selection of detection models (e.g., RetinaNet-ResNet50, FasterRCNN-ResNet50, MobileNetV3-Large-FPN)
Training configuration and hyperparameters

Algorithmic workflow (with diagrams/pseudocode)

Preprocessing and annotation generation
Model training loop for detection
Post-processing steps (bounding box adjustments, non-max suppression)

Challenges and considerations

Handling strong distortions and occlusions
GPU memory management and batch size limitations

\subsection{Results: Detection}

Presentation of experimental setup

Datasets used (Files 4, 5, and custom datasets)
Evaluation metrics (Mean IoU, MAE, detection success rates)

Numerical results and analysis

Performance comparison of different detection architectures
Impact of distortions on detection accuracy
Visualisation of detection outputs and failure cases

Discussion on detection performance

Analysis of false negatives/positives
Critical factors affecting detection success

\section{Conclusion}

Summary of methodologies and key findings in classification and detection
Critical analysis of strengths and weaknesses of approaches
Insights into factors affecting performance
Suggestions for improvements and areas for future research

Further dataset expansion (synthetic or real-world captures)
Refinement of network architectures and training strategies
Enhanced evaluation techniques and theoretical analysis

\printbibsection

\appendices

\renewcommand{\thesection}{\Alph{section}}

\section{Detailed Network Architecture}

Appendix 1

\section{Classification Network}

Appendix 2

\end{document}
