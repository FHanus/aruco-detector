\documentclass[conference]{IEEEtran}
\IEEEoverridecommandlockouts

\usepackage{algorithm}
\usepackage{algorithmic}
\usepackage{float}

\input{template}

\begin{document}

\title{Deep Learning Methods for ArUco Marker Detection and Classification Under Challenging Distortions}

\author{
\IEEEauthorblockN{
    Filip Hanuš,
}
\IEEEauthorblockA{
    School of Engineering, College of Art, Technology and Environment,\\
    University of the West of England, Bristol, UK\\
    Email: filip2.hanus@uwe.ac.uk}
}

\maketitle
\begin{abstract}

Blabla

\end{abstract}

\begin{IEEEkeywords}
ArUco Markers, Deep Learning, Computer Vision, Object Detection, Image Classification, Fiducial Markers
\end{IEEEkeywords}

\section{Introduction}

ArUco markers (figure \ref{fig:aruco_marker_basic}) are square visual codes designed for easy detection and identification by computer vision algorithms. 
They are used for positioning, tracking, and navigation in various applications such as robotics, virtual reality, or industrial automation.

\begin{figure}[h]
    \centering
    \includegraphics[width=0.1\textwidth]{images/aruco-marker-1.png}
    \caption{Basic ArUco marker example}
    \label{fig:aruco_marker_basic}
\end{figure}

Some specific examples of ArUco marker usage can be seen in Figure \ref{fig:aruco_markers}. For their use in these applications, 
they need to be tracked and/or recognised, which creates the need for robust detection and classification methods.

\begin{figure}[h]
    \centering
    \begin{subfigure}[b]{0.45\textwidth}
        \centering
        \includegraphics[width=\textwidth]{images/aruco-example-1.png}
        \caption{Robot localisation (\cite{AI-Assisted-Drone-Localization})}
        \label{fig:aruco1}
    \end{subfigure}
    \hfill
    \begin{subfigure}[b]{0.45\textwidth}
        \centering
        \includegraphics[width=\textwidth]{images/aruco-example-2.png}
        \caption{Drone localisation (\cite{Nakajima2024AboutAE})}
        \label{fig:aruco2}
    \end{subfigure}
    \caption{Examples of ArUco markers in different conditions}
    \label{fig:aruco_markers}
\end{figure}

As \textcite{FiducialMarkerNoisy}, \textcite{ROMERORAMIREZ2021104094}, and others point out, traditional detection methods perform well under controlled 
conditions, but less reliably with real-world challenges, such as rotations, blur and noise. Building on that, this work explores deep learning 
approaches to solve the following challenges:

\begin{enumerate}
  \item Recognising which variation of the marker is present: Classifying ArUco markers under varying blur, noise, and rotation conditions. 
  \item Tracking marker's position in the image: Detecting ArUco markers under varying blur, noise, and rotation conditions. 
\end{enumerate}

\section{Classification}

\subsection{Classification Method}

\begin{figure}[h]
  \centering
  \begin{subfigure}[b]{0.2\textwidth}
      \centering
      \includegraphics[width=\textwidth]{images/aruco-file3-classification-1.png}
      \caption{Class 0 (out of 100) marker example 1}
      \label{fig:class_ex1}
  \end{subfigure}
  \hfill
  \begin{subfigure}[b]{0.2\textwidth}
      \centering
      \includegraphics[width=\textwidth]{images/aruco-file3-classification-2.png}
      \caption{Class 0 (out of 100) marker example 2}
      \label{fig:class_ex2}
  \end{subfigure}
  
  \vspace{0.5cm}
  
  \begin{subfigure}[b]{0.2\textwidth}
      \centering
      \includegraphics[width=\textwidth]{images/aruco-file3-classification-3.png}
      \caption{Class 1 (out of 100) marker example 1}
      \label{fig:class_ex3}
  \end{subfigure}
  \hfill
  \begin{subfigure}[b]{0.2\textwidth}
      \centering
      \includegraphics[width=\textwidth]{images/aruco-file3-classification-4.png}
      \caption{Class 1 (out of 100) marker example 2}
      \label{fig:class_ex4}
  \end{subfigure}
  \caption{Examples of ArUco marker classification under different conditions (from the provided dataset)}
  \label{fig:classification_examples}
\end{figure}

The classification dataset examples are visible in figure \ref{fig:classification_examples}.
There are 100 variations/classes of the ArUco marker, and the development of a suitable classification model was broken down into the following steps:

\begin{enumerate}
  \item Create a data plotter for datasets and results
  \item Develop data augmentation script for prepation of a larger dataset  
  \item Prepare classification network architectures
  \item Design experiment to compare classification architectures and parameters
  \item Train and test the final selected classification model
\end{enumerate}

\subsubsection{Data Preparation}

Firstly, the dataset browser was developed. It was initially used to browse the dataset and understand the data. It shows the image, pixel distribution, and basic dataset statistics.
The idea was that this same tool would later be used to plot the data and results. This can be seen in Figure \ref{fig:data_browser_1}.

\begin{figure}[h]
  \centering
  \includegraphics[width=0.45\textwidth]{images/aruco-dataset-browser-1.png}
  \caption{Dataset browser used with the File2 dataset}
  \label{fig:data_browser_1}
\end{figure}

After analysing the data, the data augmentation script was developed. It was used to create a larger dataset for training.
Trial and error was used to find the optimal parameters for the data augmentation that would approximate the augmentations performed
on the data in the dataset provided in File3. With this, the dataset could easily be expanded to include any number of images. The selected
dataset size was 1000 images per class, each image being 512x512 pixels. The data augmentation pipeline uses the raw, unprocessed images in File1, in this file, 
there is a total of 100 images, each is a different variation of the ArUco marker. All of the parameters can be changed in the central configuration file.

The methodology of the data augmentation pipeline is shown below:

\begin{figure}[H]
  \begin{algorithm}[H]
  \caption{Data Augmentation Pipeline}
  \begin{algorithmic}[1]
  \STATE List all of the images in the File1 dataset
  \WHILE{size of augmentations list $<$ target dataset size}
      \STATE Select random rotation, between 0 and 360 degrees
      \STATE Select random scaling, between 50\% and 100\% of original size
      \STATE Select random blur, between 0 and 14 sigma
      \STATE Select random noise, between 0\% and 10\% of pixels
      \STATE Append the set of augmentations to the list
    \ENDWHILE
  \FOR{each image in File1}
      \STATE Create folder named after the image's number (0-99)
      \STATE Read the raw image
      \FOR{each set in the list of augmentations}
        \STATE Apply the augmentations to the image
        \STATE Save the augmented image to the new folder
      \ENDFOR
    \ENDFOR
  \end{algorithmic}
  \end{algorithm}
  \caption{Classification training workflow}
  \end{figure}

\subsubsection{Network Architectures}

Multiple architectures were implemented from the PyTorch library, and tested. To begin with, based on \textcite{pytorch_cifar10}, a simple CNN was implemented to test
how a simple model would perform. Then AlexNet with and without pre-training was implemented to compare the difference. ResNet18 and GoogLeNet were also implemented
from the guidance within the documentation \textcite{pytorch_models}. The modularity of the code allows further model adoption quickly.

Out of many more trialed architectures, the following ones functioned straight out of the box and were used in the initial experiment:

\begin{itemize}
    \item \textbf{MinimalCNN}: A lightweight custom architecture with:
    \begin{itemize}
        \item Three convolutional blocks (32$\rightarrow$64$\rightarrow$128 features)
        \item Batch normalisation and dropout for regularisation
        \item Global average pooling for dimension reduction
    \end{itemize}
    
    \item \textbf{AlexNet Variants}:
    \begin{itemize}
        \item Clean implementation without pre-training
        \item Version with pre-trained features
    \end{itemize}
    
    \item \textbf{ResNet18}: Pre-trained implementation
    \item \textbf{GoogLeNet}: Pre-trained implementation
\end{itemize}

\subsubsection{Training Configuration}

Combinations of all of the selected networks, and all of the following additional parameters were trained:

\begin{itemize}
  \item \textbf{Datasets}: File3 (provided dataset with 100 images in all of the 100 provided classes, strong distortions)
  \item \textbf{Models}: MinimalCNN, AlexNet-clean, AlexNet, ResNet18, GoogLeNet
  \item \textbf{Batch Sizes}: 32, 64
  \item \textbf{Additional Random Training Transformations}: None, random rotations, random rotation blur and noise
\end{itemize}

These experiment parameters form 30 different combinations, thus 30 different networks and sets of results, with training time on a single GPU for total of nearly 8 hours. 
With a single change in the configuration file, the entire experiment can be changed, the custom dataset (or more datasets) can be used (increasing time per epoch), different
batch sizes can be used (increasing memory usage beyond the used GPU's capability), different transformations can be used, and different models can be used.

Additional settings and parameters purely for the purpose of limiting the time to run the experiment were used:

\begin{itemize}
    \item \textbf{Loss Function}: Cross-entropy
    \item \textbf{Optimizer}: Adam with learning rate 3e-4
    \item \textbf{Training Duration}: 50 epochs with early stopping at 96\% training accuracy
\end{itemize}

After the experiments were completed, the results were plotted and analysed. The results are shown in the following section, and full results are available in the results directory
and the appendices.

The final model was trained on the custom dataset, with the following parameters and settings derived based on the best performing combination of the initial experiments:

\begin{itemize}
  \item \textbf{Datasets}: Custom dataset (1000 images per class)
  \item \textbf{Model}: GoogLeNet
  \item \textbf{Batch Sizes}: 32
  \item \textbf{Additional Random Training Transformations}: None
\end{itemize}

\subsubsection{Method Implementation}

The experiment implementation followed this algorithmic workflow:

\begin{figure}[H]
\begin{algorithm}[H]
\caption{Classification Experiment Pipeline}
\begin{algorithmic}[1]
\STATE Load configuration for dataset, model, batch size, and transformations
\FOR{each combination in configurations}
    \STATE Load dataset
    \STATE Setup results directory
    \STATE Initialise model and transforms
    \STATE Create data loaders
    \STATE Initialise loss function, optimiser
    \STATE Train and validate model for number of epochs
    \STATE Plot the training progress
    \STATE Run evaluation on test set
    \STATE Plot classification metrics
    \STATE Save all results
\ENDFOR
\end{algorithmic}
\end{algorithm}
\caption{Classification experiment workflow}
\end{figure}

The model training loop is shown below:

\begin{figure}[H]
\begin{algorithm}[H]
\caption{Classification Training and Validation Pipeline}
\begin{algorithmic}[1]
\FOR{each epoch in total number of epochs}
    \STATE Set model to train mode
    \FOR{each batch in train loader}
        \STATE Reset gradients
        \STATE Forward pass through network
        \STATE Calculate cross-entropy loss
        \STATE Backpropagate and update weights
        \STATE Add loss and accuracy to totals
    \ENDFOR
    \STATE Calculate average training loss and accuracy
    \STATE Set model to evaluation mode
    \FOR{each batch in validation loader}
        \STATE Forward pass through network
        \STATE Calculate cross-entropy loss
        \STATE Add loss and accuracy to totals
    \ENDFOR
    \STATE Calculate average validation loss and accuracy
    \STATE Save model if the best validation accuracy was achieved
    \STATE Check for early stopping criteria
\ENDFOR
\end{algorithmic}
\end{algorithm}
\caption{Classification training workflow}
\end{figure}

The final training loop reused most of the code from the above. The only change was that
after successful training on the custom dataset, the model was loaded again, and the two original
datasets were used to create test only dataloaders and the model was tested on them individually.

\subsection{Classification Results}

The classification experiments were conducted in two phases. Initially, multiple architectures
and configurations were tested on the File3 dataset to determine the optimal approach. Subsequently, the best
performing configuration was trained on a custom dataset and evaluated against both File2 and File3.

\subsubsection{Initial Experiments}

The experiments encompassed various architectures with different batch sizes and data augmentation strategies. Each
configuration was tested on 1000 samples from File3, with the following results:

\begin{itemize}
    \item \textbf{AlexNet (Clean Implementation)}:
        \begin{itemize}
            \item Batch size 64, rotation (training time) augmentation: 98.60\% accuracy (F1: 98.51\%)
        \end{itemize}

    \item \textbf{AlexNet (Clean Implementation)}:
        \begin{itemize}
            \item Batch size 64, rotation (training time) augmentation: 98.60\% accuracy (F1: 98.51\%)
        \end{itemize}
    
    \item \textbf{AlexNet (Pre-trained)}:
        \begin{itemize}
            \item Batch size 64, full (training time) augmentation: 73.70\% accuracy (F1: 72.15\%)
        \end{itemize}
    
    \item \textbf{ResNet18}:
        \begin{itemize}
            \item Batch size 32, no (training time) augmentation: 100.00\% accuracy (F1: 100.00\%)
        \end{itemize}
    
    \item \textbf{GoogLeNet}:
        \begin{itemize}
            \item Batch size 32, no (training time) augmentation: 99.90\% accuracy (F1: 99.87\%)
        \end{itemize}
\end{itemize}

Several key trends emerged from these results. The clean AlexNet implementation surprisingly outperformed
its pre-trained counterpart, suggesting that the ImageNet pre-training might not transfer well to ArUco marker classification. This could be
due to the fact that the pre-trained model was trained on a different dataset, and the ArUco markers are not very similar to the ImageNet dataset. Alternatively,
the implementation of the AlexNet model with the weights from PyTorch was not done aproproately, and the model was not able to train effectively.
ResNet18 achieved perfect accuracy without augmentation, though its performance slightly decreased with additional augmentations.
GoogLeNet maintained consistently high performance across all configurations, never dropping below 99.50\% accuracy.
Retrospectively, the ResNet18 model was the best performing model, achieving perfect accuracy on the File2 dataset, and 97.48\% accuracy on the File3 dataset.

The results of the minimal CNN are as expected, although the model was indeed training and given many more epochs, it would likely achieve better results. Due to
the lack of available GPU resources, and time, this was not attempted. During the initial development, this model was being tested on regularly 
to validate the functionality, and when tested thoroughly (such as in the commit d9fd8ad), the model has achieved 92.30\% validationaccuracy after
100 epochs of training. To solidify these results, this experiment would have to be repeated with the current codebase and for all of the combinations to
ensure the same and unbiased results. 

\subsubsection{Final Model Performance}

Based on the previous findings, GoogLeNet was selected for the final model due to its consistent performance and efficient training
characteristics. While ResNet18 achieved marginally better results in some configurations, GoogLeNet showed more stable performance
across different parameters and required less computational resources. The model was trained on FileCustom1, a synthetic dataset created
to approximate the conditions in Files 2 and 3.

The final evaluation yielded these results:
\begin{itemize}
    \item \textbf{Training Evaluation} (2,500 samples):
        \begin{itemize}
            \item Accuracy: 100.00\%
            \item Precision: 100.00\%
            \item Recall: 100.00\%
            \item F1-score: 100.00\%
        \end{itemize}
    \item \textbf{File2 Evaluation} (10,000 samples):
        \begin{itemize}
            \item Accuracy: 100.00\%
            \item Precision: 100.00\%
            \item Recall: 100.00\%
            \item F1-score: 100.00\%
        \end{itemize}
    \item \textbf{File3 Evaluation} (10,000 samples):
        \begin{itemize}
            \item Accuracy: 97.48\%
            \item Precision: 97.68\%
            \item Recall: 97.48\%
            \item F1-score: 97.51\%
        \end{itemize}
\end{itemize}

These results demonstrate the model's robust performance across different distortion conditions. The perfect scores on File2 indicate excellent handling of weak distortions, while the strong performance on File3 (misclassifying only 252 out of 10,000 samples) shows good resilience to more challenging conditions.

\section{Detection}

The detection problem was broken down into the following steps:

\begin{enumerate}
  \item Adapt and prepare the data plotter for detection datasets and results
  \item Develop data augmentation script
  \item Prepare detection network architectures
  \item Design experiment to compare detection architectures and settings
  \item Train and test the final detection model
\end{enumerate}


\begin{figure}[h]
  \centering
  \includegraphics[width=0.5\textwidth]{images/aruco-dataset-browser-2.png}
  \caption{Dataset browser used with the File4 dataset}
  \label{fig:data_browser_2}
\end{figure}

\begin{figure}[h]
  \centering
  \begin{subfigure}[b]{0.2\textwidth}
      \centering
      \includegraphics[width=\textwidth]{images/aruco-file3-detection-1.png}
      \caption{Detection example 1}
      \label{fig:det_ex1}
  \end{subfigure}
  \hfill
  \begin{subfigure}[b]{0.2\textwidth}
      \centering
      \includegraphics[width=\textwidth]{images/aruco-file3-detection-2.png}
      \caption{Detection example 2}
      \label{fig:det_ex2}
  \end{subfigure}
  
  \vspace{0.5cm}
  
  \begin{subfigure}[b]{0.2\textwidth}
      \centering
      \includegraphics[width=\textwidth]{images/aruco-file3-detection-3.png}
      \caption{Detection example 3}
      \label{fig:det_ex3}
  \end{subfigure}
  \hfill
  \begin{subfigure}[b]{0.2\textwidth}
      \centering
      \includegraphics[width=\textwidth]{images/aruco-file3-detection-4.png}
      \caption{Detection example 4}
      \label{fig:det_ex4}
  \end{subfigure}
  \caption{Examples of ArUco marker detection under different conditions (from the provided dataset)}
  \label{fig:detection_examples}
\end{figure}

\subsection{Method: Detection}

Overview of detection approach

Data preparation: merging ArUco markers with office images
Selection of detection models (e.g., RetinaNet-ResNet50, FasterRCNN-ResNet50, MobileNetV3-Large-FPN)
Training configuration and hyperparameters

Algorithmic workflow (with diagrams/pseudocode)

Preprocessing and annotation generation
Model training loop for detection
Post-processing steps (bounding box adjustments, non-max suppression)

Challenges and considerations

Handling strong distortions and occlusions
GPU memory management and batch size limitations

\subsection{Results: Detection}

Presentation of experimental setup

Datasets used (Files 4, 5, and custom datasets)
Evaluation metrics (Mean IoU, MAE, detection success rates)

Numerical results and analysis

Performance comparison of different detection architectures
Impact of distortions on detection accuracy
Visualisation of detection outputs and failure cases

Discussion on detection performance

Analysis of false negatives/positives
Critical factors affecting detection success

\section{Conclusion}

Summary of methodologies and key findings in classification and detection
Critical analysis of strengths and weaknesses of approaches
Insights into factors affecting performance
Suggestions for improvements and areas for future research

Further dataset expansion (synthetic or real-world captures)
Refinement of network architectures and training strategies
Enhanced evaluation techniques and theoretical analysis

\printbibsection

\appendices

\renewcommand{\thesection}{\Alph{section}}

\section{Detailed Network Architecture}

Appendix 1

\section{Classification Network}

Appendix 2

\end{document}
